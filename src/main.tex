\documentclass{UoYCSproject}
\addbibresource{dummyBib.bib}
\author{Lilian Blot}
\title{An example of a project reports in \LaTeXe\ with the   \textsf{UoYCSproject} class}
\date{Version 3.0, 2020-November}
\supervisor{Jeremy L. Jacob}
\BSc

\dedication{To my son}

\acknowledgements{
    I would like to thank my goldfish for all the help it gave me
    writing this document.

    As usual, my boss was an inspiring source of sagacious advice.
}

% Document

\begin{document}
    \pagenumbering{roman}
    \maketitle
    \listoffigures
    \listoftables
%\renewcommand*{\lstlistlistingname}{List of Listings}
%\lstlistoflistings

\chapter{Introduction}
\label{cha:Introduction}
Introduction
Late delivery of software projects remains an ever present challenge, often attributed to the inherent uncertainties and complexities associated with development. There are multiple factors that affect the delivery of a software project, however it is clear that project planning plays a role in this [45].
Agile Software development is a type of iterative planning method where a project is broken up into smaller “user stories” or tickets, each defining a specific task or part of a project that must be completed. Generally each ticket has an estimate attached to it, defining the time or effort it will take to finish. Some projects use time-based estimates like hours or days to estimate tickets, others teams assign “Story Points”, an abstract unit of perceived complexity, risk, and effort involved in implementing a specific feature or piece of functionality [Book]. Estimates enable a Team leader to plan what tickets can be completed in a “sprint”, or “iteration”, usually a period of 2-4 weeks, based on how many points they think the team will be able to do [Agile planning and development methods]. The process for estimating tickets can be time consuming, the faster that a ticket is estimated, the faster it can be planned for completion. Furthermore, the more accurate an estimation is, the more likely it is that a release, which in agile, is a collection of tickets associated with features or bug-fixes in a release, can be delivered on time, as the work will have been planned accordingly.
When a ticket is created, it is placed into a to-do list of tickets called a “backlog”. Tickets in the backlog are then estimated for their effort or time to complete, and are then ordered according to multiple factors including this estimate and the priority of the ticket. Generally in projects, work cannot be carried out on a ticket until it has been estimated. This emphasises the importance of both the speed and accuracy of ticket estimation in the successful management of projects.
Automated ticket estimation may enable the correct prioritisation of tickets before any estimation activities are carried out by the team or members of the team.
By using a large language model, the information provided in a tickets title and description can be used as features in a classification neural network.


\chapter{Background}
\label{cha:background}
Lorem ipsum dolor sit amet, consectetur adipiscing elit. Pellentesque
quis quam at nisi iaculis aliquet vel et quam. Aliquam er

\chapter{Methodology}
\label{cha:methodology}
Lorem ipsum dolor sit amet, consectetur adipiscing elit. Pellentesque
quis quam at nisi iaculis aliquet vel et quam. Aliquam era

    5hiw i3fe
\printbibliography

\end{document}